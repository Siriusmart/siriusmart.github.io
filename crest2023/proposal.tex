\documentclass{article}
\usepackage[utf8]{inputenc}
\usepackage{geometry}
\title{Stem Biology}
\author{Sirius Chan}
\begin{document}
\maketitle

\newpage

\section{Hypothesis}

The aim of the investigation is to find out how fruit flies reacts to different environments, specifically the lack of protein and/or glucose source.

\noindent\\
The hypothesis is that when not enough protein and/or glucose is given, fruit flies would priortise reproduction over its own survival, because without reproduction, all fruit flies will die out, and they should be natrually hardwired to do so.

\noindent\\
If the hypothesis is true, then reproduction should not decline as quick as survival rate when the amount of protein/glucose provided decreases.

\noindent\\
Although an abundent amount of both protein and glucose are preferred for them to survive and reproduce, fruit flies may react differently to the lack of each substance. For example, a lack of protein may inhibits reproduction more (because it is related to growth) while glucose affects survival rate more (as it is more related to day to day metabolism).

\section{Investigation plan}
\subsection{Preparing the veils}

From to Dr.\ Bass the formula for fruit fly media mix (food) is, the formula uses sucrose as a source of glucose, and yeast as a source of protein.\\

{
\centering
\begin{tabular}{|c|c|}
  \hline
  Item & Amount\\
  \hline
  \hline
  Distilled water & 500ml\\
  Agar & 5g\\
  Brewers yeast & 50g\\
  Sucrose & 50g\\
  Propionic acid & 2ml\\
  Methyl 4-hydroxybenzoate & 0.5g\\
  \hline
\end{tabular}
\par
}

\noindent\\\\
(Methyl 4-hydroxybenzoate is dissolved in 2.5ml of ethanol)

\noindent\\
This creates a large batch of media mix each time, with the same amount of sucrose and yeast. However, the investigation requires small number of vials of each yeast/sucrose amount.

\noindent\\
(The sucrose/yeast solubility is more complicated than what I anticipated, so maybe as for now I should use a fixed sucrose to yeast ratio)

\noindent\\
A slight modification of how the media mix is made can be used, here's the new formula.

{
\centering
\begin{tabular}{|c|c|}
  \hline
  Item & Amount\\
  \hline
  \hline
  Distilled water & 300ml\\
  Agar & 3g\\
  Propionic acid & 1.2ml\\
  Methyl 4-hydroxybenzoate & 0.3g\\
  Yeast/sucrose solution & Varies\\
  \hline
\end{tabular}
\par
}

\begin{enumerate}
  \item Weigh out agar and place in 1 litre Simax reagent bottle.
  \item Add 200ml distilled water, leave cap loose and autoclave.
  \item Add propionic acid and methyl 4-$\dots$ in dissolved 2.5ml ethanol.
  \item Pour over sterile 30ml vials filling to approximately $\frac{1}{3}$ volume.
  \item At the same time, from 100ml of distilled water. Prepare sucrose solution of $150gdm^{-3}$, then dissolve $150gdm^{-3}$ worth of brewers yeast in the sucrose solution.
  \item Autoclave to sterilise the sucrose-yeast solution.
  \item After sterlisation, add custom amount of the solution to each vial.
  \item Add distilled water so that all vials has the same amount of mix as the 50\% extra solution vial.
  \item Place all vials in a 55\textcelsius~water path to allow equilibriate.
  \item Cap vials and store in fridge, wait for it to set.
\end{enumerate}

\noindent\\
With this procedure the same-as-original-forumla vial can be made by adding the sucrose/yeast solution with volume half the volume of solution that is already in the vial. (Basically add 50\% extra volume)

\noindent\\
Adding distilled water helps ensure that all vials have the same concentration of substances like agar and propionic acid.

\subsection{Gathering data}

Vials of various sucrose/yeast concentration (10\%, 20\%, \dots~,50\% extra) is used, this number can be plotted against the survival rate, and reproduction rate of flies living in that vial.

\noindent\\
If this is correct, not only the lines should not be straight, but there should be somewhat sharp bend downwards at the point of "not enough food". Even if this is not the case, the investigation can continue, because it doesn't really matter.

\noindent\\
For the ease of counting, a smaller amount of fruit flies should be let in each vial, but that may introduce too much random factors such as male/female numbers (fruit flies at that size can be difficult to distinguish gender), and problems with individual flies. So a decent amount, talking about 5 to 8 flies, should be used in each vial.

\noindent\\
These number should be recorded for each vial, daily:

\begin{itemize}
  \item Currently alive fruit flies
  \item Total number of fruit flies ever lived (for calculating reproduction rate)
\end{itemize}

\noindent
(This is assuming they don't become a dark blurry mess that is impossible to count)

\noindent\\
Any interesting behaviour, for example lack of activity (may happen with low sucrose concentration) should be recorded.

\noindent\\
Doing 5 groups (of 10 to 50\% extra solution) with repeats should give somewhat meaningful data. However, if there are more concentration that might be interesting, for example "turning point" where it separates enough and not enough.

\noindent\\
At this point, analysing data should not be a thing to worry, the first priority would be to record everything down.

\end{document}